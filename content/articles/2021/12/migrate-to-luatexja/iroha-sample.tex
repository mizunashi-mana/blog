\documentclass[
    luatex,
    japanese,
    unicode,
    titlepage,
    pdfusetitle
]{ltjsarticle}

\usepackage{luatexja}
\usepackage{hyperref}
\usepackage{babel}
\usepackage{bookmark}
\usepackage[no-math,haranoaji,deluxe]{luatexja-preset}
\usepackage{unicode-math}

\setmainfont{STIX Two Text}
\setmathfont{STIX Two Math}

% some math fonts not support bold
\DeclareFontShape{TU}{STIXTwoMath(1)}{b}{n}{<->ssub*STIXTwoMath(1)/m/n}{}
\DeclareFontShape{TU}{STIXTwoMath(2)}{b}{n}{<->ssub*STIXTwoMath(2)/m/n}{}
\DeclareFontShape{TU}{STIXTwoMath(3)}{b}{n}{<->ssub*STIXTwoMath(3)/m/n}{}

\begin{document}
    \pagestyle{empty}

    \begin{tabular}{ll}
        \textbackslash textup & \textup{いろはにほへとちりぬるを} \\
        \textbackslash textit & \textit{いろはにほへとちりぬるを} \\
        \textbackslash textsl & \textsl{いろはにほへとちりぬるを} \\
        \textbackslash textsc & \textsc{いろはにほへとちりぬるを} \\
        \textbackslash textmd & \textmd{いろはにほへとちりぬるを} \\
        \textbackslash textbf & \textbf{いろはにほへとちりぬるを} \\
        \textbackslash textrm & \textrm{いろはにほへとちりぬるを} \\
        \textbackslash textmc & \textmc{いろはにほへとちりぬるを} \\
        \textbackslash textsf & \textsf{いろはにほへとちりぬるを} \\
        \textbackslash textgt & \textgt{いろはにほへとちりぬるを} \\
        \textbackslash texttt & \texttt{いろはにほへとちりぬるを}
    \end{tabular}
\end{document}
