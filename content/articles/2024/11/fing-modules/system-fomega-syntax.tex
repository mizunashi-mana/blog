\documentclass[lualatex,12pt,unicode]{article}

\usepackage{luatexja-preset}
\usepackage{amsmath}
\usepackage{amssymb}
\usepackage{ascmac}
\usepackage{mathtools}
\usepackage{ebproof}

\setlength{\paperwidth}{20cm}
\setlength{\paperheight}{30cm}

% paper size
\setlength{\pdfpagewidth}{\paperwidth}
\setlength{\pdfpageheight}{\paperheight}
% horizontal dimensions
\setlength{\oddsidemargin}{-1in}
\setlength{\evensidemargin}{\oddsidemargin}
\setlength{\textwidth}{\paperwidth}
% vertical dimension
\setlength{\topmargin}{-1in}
\setlength{\headheight}{0pt}
\setlength{\headsep}{0pt}
\setlength{\topskip}{0pt}
\setlength\footskip{0pt}
\setlength{\textheight}{\paperheight}


\begin{document}
\pagestyle{empty}

\begin{align*}
    \begin{array}{rcll}
        \kappa
        &\Coloneqq &\star &(\text{型の種}) \\
        &\mid &\kappa_1 \to \kappa_2 &(\text{型構築子の種}) \\
        \tau
        &\Coloneqq &t &(\text{型変数}) \\
        &\mid &\tau_1 \to \tau_2 &(\text{関数型}) \\
        &\mid &\{\overrightarrow{l: \tau}\} &(\text{レコード型}) \\
        &\mid &\forall t: \kappa\ldotp \tau &(\text{多相型}) \\
        &\mid &\exists t: \kappa\ldotp \tau &(\text{存在型}) \\
        &\mid &\lambda t: \kappa\ldotp \tau &(\text{型構築子}) \\
        &\mid &\tau_1\; \tau_2 &(\text{型構築}) \\
        e
        &\Coloneqq &x &(\text{変数}) \\
        &\mid &\lambda x: \tau\ldotp e &(\text{ラムダ式}) \\
        &\mid &e_1\; e_2 &(\text{適用}) \\
        &\mid &\{\overrightarrow{l = e}\} &(\text{レコード式}) \\
        &\mid &e.l &(\text{レコードアクセス}) \\
        &\mid &\Lambda t: \kappa\ldotp e &(\text{多相化}) \\
        &\mid &e\; \tau &(\text{具象化}) \\
        &\mid &{\mathrm{pack}\langle \tau_1, e\rangle}_{\tau_2} &(\text{存在量化}) \\
        &\mid &\mathrm{unpack}\langle t: \kappa, x: \tau\rangle = e_1\ldotp e_2 &(\text{存在アクセス}) \\
        \Gamma
        &\Coloneqq &t: \kappa \\
        &\mid &x: \tau \\
        &\mid &\epsilon \\
        &\mid &\Gamma_1, \Gamma_2
    \end{array}
\end{align*}
\end{document}
