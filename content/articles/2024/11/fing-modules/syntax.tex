\documentclass[12pt]{article}

\usepackage{luatexja-preset}
\usepackage{amsmath}
\usepackage{amssymb}
\usepackage{ascmac}
\usepackage{mathtools}
\usepackage{ebproof}

\setlength{\paperwidth}{20cm}
\setlength{\paperheight}{30cm}

% paper size
\setlength{\pdfpagewidth}{\paperwidth}
\setlength{\pdfpageheight}{\paperheight}
% horizontal dimensions
\setlength{\oddsidemargin}{-1in}
\setlength{\evensidemargin}{\oddsidemargin}
\setlength{\textwidth}{\paperwidth}
% vertical dimension
\setlength{\topmargin}{-1in}
\setlength{\headheight}{0pt}
\setlength{\headsep}{0pt}
\setlength{\topskip}{0pt}
\setlength\footskip{0pt}
\setlength{\textheight}{\paperheight}


\begin{document}
\pagestyle{empty}

\begin{align*}
    \begin{array}{rcll}
        X
        &\Coloneqq &\cdots &(\text{変数}) \\
        K
        &\Coloneqq &\cdots &(\text{種}) \\
        T
        &\Coloneqq &\cdots \mid P &(\text{型}) \\
        E
        &\Coloneqq &\cdots \mid P &(\text{式}) \\
        P
        &\Coloneqq &M &(\text{パス}) \\
        M
        &\Coloneqq &X \\
        &\mid &\{B\} &(\text{定義群}) \\
        &\mid &M.X &(\text{射影}) \\
        &\mid &\mathrm{fun}\;X: S \Rightarrow M &(\text{ファンクタ}) \\
        &\mid &X_1\; X_2 &(\text{ファンクタへの適用}) \\
        &\mid &X \triangleright S &(\text{シーリング}) \\
        B
        &\Coloneqq &\mathrm{val}\; X = E &(\text{値定義}) \\
        &\mid &\mathrm{type}\; X = T &(\text{型定義}) \\
        &\mid &\mathrm{module}\; X = M &(\text{モジュール定義}) \\
        &\mid &\mathrm{sig}\; X = S &(\text{シグネチャ定義}) \\
        &\mid &\mathrm{include}\; M &(\text{モジュール展開}) \\
        &\mid &\epsilon &(\text{空}) \\
        &\mid &B_1; B_2 &(\text{結合}) \\
        S
        &\Coloneqq &P &(\text{シグネチャパス}) \\
        &\mid &\{D\} &(\text{宣言群}) \\
        &\mid &(X: S) \rightarrow S &(ファンクタシグネチャ) \\
        &\mid &S\;\mathrm{where}\;\mathrm{type}\;\overrightarrow{X} = T &(\text{制約付きシグネチャ}) \\
        D
        &\Coloneqq &\mathrm{val}\; X: T &(\text{値宣言}) \\
        &\mid &\mathrm{type}\; X: K &(\text{型宣言}) \\
        &\mid &\mathrm{type}\; X = T &(\text{型定義}) \\
        &\mid &\mathrm{module}\; X: S &(\text{モジュール宣言}) \\
        &\mid &\mathrm{sig}\; X = S &(\text{シグネチャ定義}) \\
        &\mid &\mathrm{include}\; S &(\text{シグネチャ展開}) \\
        &\mid &\epsilon &(\text{空}) \\
        &\mid &D_1; D_2 &(\text{結合})
    \end{array}
\end{align*}
\end{document}
